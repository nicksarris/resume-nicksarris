\documentclass[letterpaper]{resume_cv}

\cvname{Nick Sarris}
\cvdate{May 12, 1998}
\cvaddress{ngs5st@virginia.edu}
\cvnumberphone{757-334-0477}
\cvsite{github.com/nicksarris}

\begin{document}

\Education{
University of Virginia - School of Engineering and Applied Sciences : Majoring in CS with a focus in Software Engineering / Machine Learning - Graduation Date: May 2020}

\Skills{
\textit{Languages/Toolkits:} \newline
Python, Flask, Django, HTML, CSS, Javascript, React.js, React-Native, Redux, Webpack, Node.js, MongoDB, AWS, Express, REST, JQuery, C++, SQL (Postgresql, MYSQL, etc), Bash, Latex, Solidity, Firebase, C#, Java
\newline\newline
\textit{Software:} \newline
Microsoft Office (Word, Excel, etc), Linux, Mathematica, Matlab, CAD modeling (AutoCAD), Unity, Github (Gitkraken, etc), Various IDEs (Pycharm, Eclipse, Atom, etc)}

\Classes{
\textit{Computer Science:} \newline
CS2102: Discrete Math, CS2110: Software Development Methods, CS2150: Program and Data Representation, CS3102: Theory of Computation, CS3240: Advanced Software Development, CS3330: Computer Architecture, CS4102: Algorithms, CS4414: Operating Systems, CS4501: Machine Learning, CS4630: Defense Against the Dark Arts, CS4720: Mobile Application Development, CS4753: Electronic Commerce Technology
\newline\newline
\textit{Math/Sciences:} \newline
APMA1110: Single Variable Calculus II, APMA2120: Multivariate Calculus, APMA2130: Ordinary Differential Equations, APMA3080: Linear Algebra, APMA3100: Probability, CHEM1610: Chemistry I for Engineers, PHYS1425: Physics I, PHYS2415: Physics II}

\makeprofile

\vspace{-0.7em}
\section{Technical Experience}
\vspace{1.25em}

\begin{twenty} 

	\twentyitem{}{Blockchain Developer (Stipend, Feb '18 - Mar '20)}{}{
	\begin{description}[wide, labelwidth=!, labelindent=0pt]
        \item[$\bullet$] Co-founded and launched the Stipend cryptocurrency
        \item[$\bullet$] Leveraged existing blockchain code to develop the blockchain and wallets currently integrated and running smoothly on the Stipend network of more than five thousand computers.
        \item[$\bullet$] Facilitated the growth and development of Stipend into a currency that peaked with a market cap of greater than \$15,000,000 in May 2018.
        \item[$\bullet$] Created API bindings to facilitate easy access to the Stipend blockchain through Node.js
        \item[$\bullet$] Personally developed and launched Stipend\textquotesingle s explorer, a service that allows any user to analyze the established blockchain in-depth, on AWS\textquotesingle s EC2 platform. 
        \item[$\bullet$] Leveraged Stipend\textquotesingle s blockchain to create a service that allows for a user to dispense ``paper-wallets'', printed wallets that allow for offline transfer of funds.
        \item[$\bullet$] Worked alongside three other developers to code the Stipend platform, a scalable responsive, freelancing web application built on top of the Stipend blockchain
	\end{description}}

    \twentyitem{}{Technical Intern (Northrop Grumman, Jun '19 - Aug '19)}{}{
	\begin{description}[wide, labelwidth=!, labelindent=0pt]
        \item[$\bullet$] Worked on a team of developers under an AGILE development process for Northrop Grumman\textquotesingle s TARGET IRAD
        \item[$\bullet$] Wrote extensive BASH scripts for purposes of testing existing micro-services on TARGET\textquotesingle s servers
        \item[$\bullet$] Personally developed and implemented a REST API leveraging my own modified BERT-base model, fine-tuned on an expansive dataset of CNN news articles for purposes of document summarization. 
        \item[$\bullet$] Containerized the API with Docker, ensuring easy transfer for use on cloud servers.
        \item[$\bullet$] Presented my work to the deputy director of the Office of Naval Intelligence and sat in on numerous client meetings where my work was discussed extensively.
    \end{description}}

	\twentyitem{}{Technical  Intern (Northrop Grumman, Jun '18 - Aug '18)}{}{
	\begin{description}[wide, labelwidth=!, labelindent=0pt]
    	\item[$\bullet$] Worked on a team of developers under an AGILE development process for Northrop Grumman\textquotesingle s COBRA IRAD
        \item[$\bullet$] Used Doc2Vec and other Machine Learning (NLP) techniques to classify and score blog articles found on the Internet, exploring how similar the content was to cybersecurity TTPs (Tactics, Techniques, and Procedures) found on MITRE\textquotesingle s ATT\&CK table.
        \item[$\bullet$] Created a GUI using PyQT and the aforementioned model that mirrored MITRE’s ATT\&CK table for the purposes of generating heatmaps for each individual blog article fed into the service.
        \item[$\bullet$] Used Flask to modify the existing GUI and turn it into a web application/REST API for easy access and further development.
    \end{description}}

	\twentyitem{}{Competitor (Kaggle, Current)} {}{
	\begin{description}[wide, labelwidth=!, labelindent=0pt]
    	\item[$\bullet$]For the last five years, I've been competing in various Machine Learning competitions on Kaggle, learning all I can throughout. Kaggle is a site where companies sponsor competitions for anyone to participate in.\\\\
    	Results:
    	\item[$\bullet$]Jun '19 - Jigsaw Unintended Bias in Toxicity (45th / 3165)
    	\item[$\bullet$]Jun '16 - Expedia Hotel Recommendations (89th / 1974)
    	\item[$\bullet$]Feb '17 - Allstate Claims Severity (114th / 3055)
    	\item[$\bullet$]Jun '16 - Predicting Red Hat Business Value (108th / 2271)
    	\item[$\bullet$]Apr '16 - Home Depot Product Search Relevance (137th / 2124)
    \end{description}}

	\twentyitem{}{Technical  Intern (NASA Langley Research Center, Jun '15 - Aug '15)}{}{
	\begin{description}[wide, labelwidth=!, labelindent=0pt]
    	\item[$\bullet$] Worked on a team of developers under an AGILE development process.
    	\item[$\bullet$] Used Python to code a program that implemented Machine Learning (Image Processing) techniques to analyze/scan carbon-fiber sheets for imperfections after they had been hurled at a wall, testing their structural integrity for later use.
    \end{description}}

\end{twenty}

\end{document} 
